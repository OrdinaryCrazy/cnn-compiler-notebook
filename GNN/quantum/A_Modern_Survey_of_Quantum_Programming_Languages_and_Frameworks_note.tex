% !TEX program = xelatex
\documentclass[]{ctexart}
\usepackage{geometry}
% \usepackage{fontspec}
% \setmainfont{inconsolata}
\geometry{left=1.5cm,right=1.5cm,top=2cm,bottom=2cm}
\CTEXsetup[format={\bfseries}]{section}

\title{
    A Modern Survey of Quantum Programming Languages and Frameworks\\
    Review
}
\author{
    Jingtun ZHANG 张劲暾
}
%=========================================================================
\begin{document}
    \maketitle
    \setcounter{secnumdepth}{0} %% no numbering
%=========================================================================
    \section{A. Paper summary}
    This paper try to propose a modern survey of Quantum Computing (QC) which:
    \begin{enumerate}
        \item Summarizing various aspects of quantum computing.
        \item Emphasizing the advances in quantum computing programming language and compilation.
        \item Focusing on the work in the last decade.
    \end{enumerate}
    \subsection{Content list:}
        \subsubsection*{Introduction}
            \begin{itemize}
                \item Tremendous advances in QC
                \item Compiler and Programming language contribution
                \item Nowadays Quantum Programming Languages: haven't found the proper equivalent to a 
                program statement of more traditional programming languages.
                \item QPLs with abstractions and semantics near or at the level of the application domain are necessary.
                \item Current proposed languages require further refinement before they can fully support QC algorithmic:
                \begin{enumerate}
                    \item Extremely low-level and do not leverage quantum application domain knowledge.
                    \item Higher but can not offer abstractions and data types that best support correct and productive QC programming
                \end{enumerate}
            \end{itemize}
        \subsubsection*{Background}
            \begin{itemize}
                \item High-performance application-specific accelerators, capable of making some currently-intractable
                problems tractable. 
                \item Qubits, superposition, measurement, entanglement, no-cloning.
                \item Quantum state evolvement in time with unitary operators (program)
                \item Compiler find an efficient circuit that is equivalent to input matrix by decomposing it into a 
                sequence of simpler/smaller matrices. Obstacles:
                    \begin{enumerate}
                        \item The size and the structure of the matrix largely determines the number of factors.
                        \item Decoherence: hard to deeper circuit and longer run time.
                        \item Must-made approximation
                    \end{enumerate}
            \end{itemize}
        \subsubsection*{Quantum Technologies}
            \begin{itemize}
                \item Decoherence problem
                \item Quantum Error Correction (QEC) can not totally solve the decoherence problem and resource intensive.
            \end{itemize}
        \subsubsection*{Quantum Compilation Phases}
            \begin{itemize}
                \item Typical quantum compilation flow
            \end{itemize}
        \subsubsection*{Circuit Synthesis (for unitary operators)}
            \begin{itemize}
                \item Householder: from a unitary matrix to its triangular form by 2(n-1) square roots
                \item Deutsch: almost all n-bit gates, n > 2, can represent any computing task
                \item Barenco: universal unitary operations
                \item \dots\dots
            \end{itemize}
        \subsubsection*{Quantum Programming Languages}
            \begin{itemize}
                \item OpenQASM: users declare classical and quantum registers, and apply gates to different
                qubits to perform the task.
                \item Scaffold: raise the level of abstraction to assembly level
                \item Q\#: extension for C\#
                \item ProjectQ: extension for Python
                \item Still is a substantial semantic gap between the quantum domain and the state-of-the-art in quantum language.
            \end{itemize}
        \subsubsection*{Quantum Compiler Frameworks}
            \begin{itemize}
                \item Nowadays research has largely concentrated to extending classical imperative languages to enable
                the manipulation of quantum gates.
            \end{itemize}
%=========================================================================
    \section{B. List of main strengths of the paper}
        \begin{itemize}
            \item Given out an overview of almost all aspects of Quantum Computing
            \item Described many vital research advances in the last decade
            \item Detailed concrete idea of these works
        \end{itemize}
%=========================================================================
    \section{C. List of major weaknesses of the paper}
        \begin{itemize}
            \item Without a clear outline for most sections, especially in section 5 and section 7
            \item Just stacking the brief summary of the mentioned works without discussing their key idea, influence, 
            reserved problems and relationship between each other
            \item Did not show the evolvement of Quantum Computing and difference between long-past works and modern
            important works
            \item Did not give out the general definition of problems, basical knowledge and solution ideas in every
            section, which confused me when I read stacked detail of the mentioned works
        \end{itemize}
%=========================================================================
    \section{D. Overall merit}
    \noindent Choices:\\
    \hspace*{2em}A. Good paper, I will champion it\\
    \hspace*{2em}B. OK paper, but I will not champion it\\
    \hspace*{2em}C. Weak paper, though I will not fight strongly against it\\
    \hspace*{2em}D. Reject\\
    My choice: C
%=========================================================================
    \section{E. Reviewer expertise}
    \noindent Choices:\\
    \hspace*{2em}X. I am an expert in this area\\
    \hspace*{2em}Y. I am knowledgeable in this area, but not an expert\\
    \hspace*{2em}Z. I am not an expert; my evaluation is that of an informed outsider\\
    My choice: Z
%=========================================================================
    \section{F. How much time did you spend on reviewing the paper?}
    \noindent Choices:\\
    \hspace*{2em}1. Limited: I browsed through the paper quickly\\
    \hspace*{2em}2. Medium: I read through the paper but did not check the details\\
    \hspace*{2em}3. High: I read the paper thoroughly and carefully checked the technical details\\
    My choice: 2
%=========================================================================
    \section{G. Short summary of your review explaining your overall merit score.}
        A servey paper should at least give out the probelms of the domain it summarized, but in this paper I can 
    not figure out what probelm so much research try to address. In addition, this paper did not show me the envolement  
    of great ideas and crucial works in Quantum Computing, stacking of paper-summary can not give me a clear overview of 
    modern Quantum Computing advances and future working directions.
%=========================================================================
    \section{H. Novelty}
    \noindent Choices:\\
    \hspace*{2em}1. Published before or openly commercialized\\
    \hspace*{2em}2. Incremental improvement\\
    \hspace*{2em}3. New contribution\\
    \hspace*{2em}4. Surprisingly new contribution\\
    My choice: 1
%=========================================================================
    \section{I. Potential for impact}
    \noindent Choices:\\
    \hspace*{2em}1. No impact: Will not influence other efforts\\
    \hspace*{2em}2. Limited impact: Unlikely to influence other efforts\\
    \hspace*{2em}3. Moderate impact: Will influence other efforts\\
    \hspace*{2em}4. High impact: Will have a major influence on other efforts\\
    \hspace*{2em}5. Foundational: Will open new areas of inquiry\\
    My choice: 2
%=========================================================================
    \section{J. Writing quality}
    \noindent Choices:\\
    \hspace*{2em}1. Unacceptable\\
    \hspace*{2em}2. Need improvement\\
    \hspace*{2em}3. Adequate\\
    \hspace*{2em}4. Well-written\\
    \hspace*{2em}5. Outstanding\\
    My choice: 2
%=========================================================================
    \section{K. Best paper candidate}
    \noindent Do you feel that the paper is on par with or better thn the recent LCPC 
    Best Papers?\\
    Choices:\\
    \hspace*{2em}A. Yes.\\
    \hspace*{2em}B. No.\\
    My choice: B
%=========================================================================
    \section{L. Questions for authors' response}
    Specific questions that coud affect your accept/reject decision. Remember that 
    the authors have limited space and must respond to all reviewers.
    \begin{itemize}
        \item What do you think as the great idea within these modern quantum computing works ?
        \item Is there a evolving outline of these research ?
        \item What are the basical concerns of modern quantum computing ?
    \end{itemize}
%=========================================================================
    \section{M. Comments to authors}
        I think basical probelms definition and great ideas are more important than stacking modern research
    works summary for a servey paper of modern quantum computing.
%=========================================================================
    \section{N. Comments to PC}

%=========================================================================
    \section{O. Declare concurrent work}
    \noindent Are you working on the exact same problem as this submission and 
    have a similar solution?\\
    Choices:\\
    \hspace*{2em}1. No\\
    \hspace*{2em}2. Yes, work is ongoing\\
    \hspace*{2em}3. Yes, paper submitted to LCPC 2020\\
    \hspace*{2em}4. Yes, paper submitted to another venue\\
    \hspace*{2em}5. Not sure\\
    My choice: 1
%=========================================================================
    \section{P. Recommended reviewers for 2nd round}
%=========================================================================
\end{document}